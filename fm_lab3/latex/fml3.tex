\documentclass[a4paper, 12pt]{article}
\usepackage[utf8x]{inputenc}
\usepackage{cmap}
\usepackage[english, russian]{babel}
\usepackage{indentfirst}
\usepackage[left=20mm, top=20mm, right=20mm, bottom=20mm]{geometry}
\usepackage{tikz}
\usepackage{float}
\usepackage{amsmath, amsfonts, amssymb}
\usepackage{graphicx}
\usepackage{fancybox, fancyhdr}
\usepackage{hyperref}
\usepackage{listings}
\usepackage{caption}
\usepackage{subcaption}
\usepackage{xcolor}
\pagestyle{fancy}
\fancyhf{}
\fancyhead[L]{Лабораторная работа №3}
\fancyhead[R]{Частотные методы}
\fancyfoot[C]{\thepage}
\graphicspath{{images/}}
\usetikzlibrary{patterns}
\definecolor{LightGray}{gray}{0.95}
\lstdefinestyle{pycode}{
    language=Python,
    basicstyle=\footnotesize\ttfamily,
    numbers=left,
    numberstyle=\tiny\color{gray},
    stepnumber=1,
    numbersep=5pt,
    backgroundcolor=\color{LightGray},
    showspaces=false,
    showstringspaces=false,
    showtabs=false,
    tabsize=4,
    captionpos=b,
    breaklines=true,
    breakatwhitespace=false,
    frame=none,
    rulecolor=\color{black},
    linewidth=\linewidth,
    keywordstyle=\color{red}\bfseries,
    commentstyle=\color{green!40!black},
    stringstyle=\color{blue},
    escapeinside={\%*}{*)},
    xleftmargin=0pt,
    framexleftmargin=0pt,
    framexrightmargin=0pt
}
\lstset{style=pycode}
\hypersetup{
    colorlinks=true,
    linkcolor=blue,
    filecolor=magenta,      
    urlcolor=cyan,
    pdftitle={contents setup},
    pdfpagemode=FullScreen,
}
\setlength{\parskip}{1.5mm}
\setlength{\headheight}{15pt}
\setlength{\footskip}{15pt}
\allowdisplaybreaks
\DeclareMathOperator{\sinc}{sinc}
\newcommand{\frc}[2]{\raisebox{2pt}{$#1$}\big/\raisebox{-3pt}{$#2$}}

\begin{document}
    \begin{titlepage}

        \begin{center}
        \includegraphics[width=0.3\textwidth]{itmo.png} % requires itmo.png in /images folder
        \vfill
        
        Федеральное государственное автономное образовательное учреждение высшего образования
        «Национальный Исследовательский Университет ИТМО»\\
        
        \vfill
        {\large\bf ЛАБОРАТОРНАЯ РАБОТА №3}\\
        {\large\bf ПРЕДМЕТ «ЧАСТОТНЫЕ МЕТОДЫ»}\\
        {\large\bf ТЕМА «ЖЕСТКАЯ ФИЛЬТРАЦИЯ»}
        \vfill

        \begin{flushright}
            \begin{minipage}{.45\textwidth}
            {
                \hbox{Лектор: Перегудин А. А.}
                \hbox{Практик: Пашенко А. В.}
                \hbox{Студент: Румянцев А. А.}
                \hbox{Поток: ЧАСТ.МЕТ. 1.3}
                \hbox{}
                \hbox{Факультет: СУиР}
                \hbox{Группа: R3241}
            }
            \end{minipage}
        \end{flushright}
        
        \vfill
                
        Санкт-Петербург\\
        2024
        \end{center}
    \end{titlepage}
    
    \tableofcontents

    \newpage
% \end{document}
    \section{Задание 1. Жесткие фильтры}
    Зададим такие числа $a,\,t_1,\,t_2$, что $t_1<t_2$, и рассмотрим функцию $g$ такую, что
    $g(t)=a$ при $t\in[t_1,t_2]$ и $g(t)=0$ при других $t$. $$\sqsupset a=2,\ \ t_1=-1.5,\ \ t_2=2.5,\ \ g(t)=
    \begin{cases}
        2, & t\in[t_1,t_2]\\
        0, & \text{ otherwise}
    \end{cases}
    $$


    Выберем интервал времени $T=10$ и шаг дискретизации $dt=0.01$. Зададим в python массив времени $t$ от $-0.5\cdot T$ до $0.5\cdot T+dt$
    с шагом $dt$ и включим последнюю точку. Найдем список значений $g$ и зададим зашумленную версию сигнала как
    $$
    u=g+b\cdot(\text{random}(\text{len}(t))-0.5) + c\cdot \sin(d\cdot t);
    $$


    В данном задании мы выполняем жесткую фильтрацию сигнала $u$. Алгоритм следующий: находится Фурье-образ от сигнала,
    обнуляются его значения на некоторых диапазонах частот, затем сигнал восстанавливается обратным преобразованием Фурье.
    Далее строятся графики с помощью программы на языке python. Используемый код с пояснениями находится в отдельной секции.


    В задаваемом сигнале параметр $a$ отвечает за высоту, на которую поднимется часть сигнала от нуля, а $t_1 \text{ и } t_2$ -- начало
    и конец промежутка с подъемом соответственно. Таким образом, на интервале длины $t_2-t_1=2.5+1.5=4$, начиная с $t_1=-1.5$ и заканчивая $t_2=2.5$,
    на высоте $a=2$ будет находится часть от всего сигнала, который, в свою очередь, располагается на промежутке $[-0.5\cdot T,0.5\cdot T]=[-5,5]$
    длины $2\cdot T\cdot 0.5=10$. Параметры $b,\,c,\,d$ отвечают за шум, присутствующий в сигнале. Далее будут рассмотрены графики и сделаны выводы о
    влиянии каждого параметра на сам сигнал и на его результат фильтрации.


    \subsection{Убираем высокие частоты. Фильтр нижних частот}
    Возьмем параметр $c=0$. Далее действуем в соответствии с алгоритмом. Возьмем некоторый диапазон частот $[-\nu_0, \nu_0]$, на котором оставим Фурье-образ
    сигнала $u$ неизменным, а на остальных частотах обнулим его значения. Такое поведение соответствует фильтру \textit{нижних} частот, так как он пропускает
    все частоты ниже частоты среза. Построим сравнительные графики исходного и фильтрованного сигналов на некотором интервале $[t_1,t_2]$, а также модуля
    Фурье-образа исходного и фильтрованного сигналов. Исследуем влияние частоты среза $\nu_0$ и значения параметра $b$ на эффективность фильтрации.
    
    
    Далее будут приведены рисунки полученных графиков. На каждом графике подписаны выбранные значения $b,\,c,\,d,\,\nu_0$
    (хотя, при условии, что $c=0$, менять или рассматривать параметр $d$ не требуется). Также отмечена легенда -- синим цветом
    обозначается оригинальный сигнал, красным фильтрованный.


    \begin{figure}[!htb]
        \centering
        \includegraphics[scale=0.48]{1_u_flt_u_nohigh.png}
        \captionsetup{skip=0pt}
        \caption{График исходного и фильтрованного сигналов (1)}
        \label{fig:fig1}
    \end{figure}
    \begin{figure}[!htb]
        \centering
        \includegraphics[scale=0.48]{1_abs_u_U_nohigh.png}
        \captionsetup{skip=0pt}
        \caption{График модуля Фурье-образа исходного и фильтрованного сигналов (1)}
        \label{fig:fig2}
    \end{figure}
    \begin{figure}[!htb]
        \centering
        \includegraphics[scale=0.48]{2_u_flt_u_nohigh.png}
        \captionsetup{skip=0pt}
        \caption{График исходного и фильтрованного сигналов (2)}
        \label{fig:fig3}
    \end{figure}
    \begin{figure}[!htb]
        \centering
        \includegraphics[scale=0.48]{2_abs_u_U_nohigh.png}
        \captionsetup{skip=0pt}
        \caption{График модуля Фурье-образа исходного и фильтрованного сигналов (2)}
        \label{fig:fig4}
    \end{figure}
    \begin{figure}[!htb]
        \centering
        \includegraphics[scale=0.48]{3_u_flt_u_nohigh.png}
        \captionsetup{skip=0pt}
        \caption{График исходного и фильтрованного сигналов (3)}
        \label{fig:fig5}
    \end{figure}
    \begin{figure}[!htb]
        \centering
        \includegraphics[scale=0.48]{3_abs_u_U_nohigh.png}
        \captionsetup{skip=0pt}
        \caption{График модуля Фурье-образа исходного и фильтрованного сигналов (3)}
        \label{fig:fig6}
    \end{figure}
    \begin{figure}[!htb]
        \centering
        \includegraphics[scale=0.48]{5_u_flt_u_nohigh.png}
        \captionsetup{skip=0pt}
        \caption{График исходного и фильтрованного сигналов (4)}
        \label{fig:fig7}
    \end{figure}
    \begin{figure}[!htb]
        \centering
        \includegraphics[scale=0.48]{5_abs_u_U_nohigh.png}
        \captionsetup{skip=0pt}
        \caption{График модуля Фурье-образа исходного и фильтрованного сигналов (4)}
        \label{fig:fig8}
    \end{figure}
    \begin{figure}[!htb]
        \centering
        \includegraphics[scale=0.48]{4_u_flt_u_nohigh.png}
        \captionsetup{skip=0pt}
        \caption{График исходного и фильтрованного сигналов (5)}
        \label{fig:fig9}
    \end{figure}
    \begin{figure}[!htb]
        \centering
        \includegraphics[scale=0.48]{4_abs_u_U_nohigh.png}
        \captionsetup{skip=0pt}
        \caption{График модуля Фурье-образа исходного и фильтрованного сигналов (5)}
        \label{fig:fig10}
    \end{figure}
    \begin{figure}[!htb]
        \centering
        \includegraphics[scale=0.48]{11_u_flt_u_nohigh.png}
        \captionsetup{skip=0pt}
        \caption{График исходного и фильтрованного сигналов (6)}
        \label{fig:fig11}
    \end{figure}
    \begin{figure}[!htb]
        \centering
        \includegraphics[scale=0.48]{11_abs_u_U_nohigh.png}
        \captionsetup{skip=0pt}
        \caption{График модуля Фурье-образа исходного и фильтрованного сигналов (6)}
        \label{fig:fig12}
    \end{figure}
    \begin{figure}[!htb]
        \centering
        \includegraphics[scale=0.48]{12_u_flt_u_nohigh.png}
        \captionsetup{skip=0pt}
        \caption{График исходного и фильтрованного сигналов (7)}
        \label{fig:fig13}
    \end{figure}
    \begin{figure}[!htb]
        \centering
        \includegraphics[scale=0.48]{12_abs_u_U_nohigh.png}
        \captionsetup{skip=0pt}
        \caption{График модуля Фурье-образа исходного и фильтрованного сигналов (7)}
        \label{fig:fig14}
    \end{figure}
    \begin{figure}[!htb]
        \centering
        \includegraphics[scale=0.48]{13_u_flt_u_nohigh.png}
        \captionsetup{skip=0pt}
        \caption{График исходного и фильтрованного сигналов (8)}
        \label{fig:fig15}
    \end{figure}
    \begin{figure}[!htb]
        \centering
        \includegraphics[scale=0.48]{13_abs_u_U_nohigh.png}
        \captionsetup{skip=0pt}
        \caption{График модуля Фурье-образа исходного и фильтрованного сигналов (8)}
        \label{fig:fig16}
    \end{figure}
    \begin{figure}[!htb]
        \centering
        \includegraphics[scale=0.48]{8_u_flt_u_nohigh.png}
        \captionsetup{skip=0pt}
        \caption{График исходного и фильтрованного сигналов (9)}
        \label{fig:fig17}
    \end{figure}
    \begin{figure}[!htb]
        \centering
        \includegraphics[scale=0.48]{8_abs_u_U_nohigh.png}
        \captionsetup{skip=0pt}
        \caption{График модуля Фурье-образа исходного и фильтрованного сигналов (9)}
        \label{fig:fig18}
    \end{figure}
    \begin{figure}[!htb]
        \centering
        \includegraphics[scale=0.48]{7_u_flt_u_nohigh.png}
        \captionsetup{skip=0pt}
        \caption{График исходного и фильтрованного сигналов (10)}
        \label{fig:fig19}
    \end{figure}
    \begin{figure}[!htb]
        \centering
        \includegraphics[scale=0.48]{7_abs_u_U_nohigh.png}
        \captionsetup{skip=0pt}
        \caption{График модуля Фурье-образа исходного и фильтрованного сигналов (10)}
        \label{fig:fig20}
    \end{figure}
    \begin{figure}[!htb]
        \centering
        \includegraphics[scale=0.48]{6_u_flt_u_nohigh.png}
        \captionsetup{skip=0pt}
        \caption{График исходного и фильтрованного сигналов (11)}
        \label{fig:fig21}
    \end{figure}
    \begin{figure}[!htb]
        \centering
        \includegraphics[scale=0.48]{6_abs_u_U_nohigh.png}
        \captionsetup{skip=0pt}
        \caption{График модуля Фурье-образа исходного и фильтрованного сигналов (11)}
        \label{fig:fig22}
    \end{figure}
    \begin{figure}[!htb]
        \centering
        \includegraphics[scale=0.48]{10_u_flt_u_nohigh.png}
        \captionsetup{skip=0pt}
        \caption{График исходного и фильтрованного сигналов (12)}
        \label{fig:fig23}
    \end{figure}
    \begin{figure}[!htb]
        \centering
        \includegraphics[scale=0.48]{10_abs_u_U_nohigh.png}
        \captionsetup{skip=0pt}
        \caption{График модуля Фурье-образа исходного и фильтрованного сигналов (12)}
        \label{fig:fig24}
    \end{figure}
    \begin{figure}[!htb]
        \centering
        \includegraphics[scale=0.48]{9_u_flt_u_nohigh.png}
        \captionsetup{skip=0pt}
        \caption{График исходного и фильтрованного сигналов (13)}
        \label{fig:fig25}
    \end{figure}
    \begin{figure}[!htb]
        \centering
        \includegraphics[scale=0.48]{9_abs_u_U_nohigh.png}
        \captionsetup{skip=0pt}
        \caption{График модуля Фурье-образа исходного и фильтрованного сигналов (13)}
        \label{fig:fig26}
    \end{figure}


    Исходя из графиков можно сделать вывод, что значение параметра $b$ отвечает за амплитуду каждой волны.
    Чем больше значение $b$, тем зашумленнее, <<грязнее>> выглядит сигнал, так как амплитуды волн возрастают.
    Фильтрованный сигнал при больших значениях $b$ также имеет более глубокие ямы и высокие подъемы, то есть
    испытывает увеличение амплитуды. Наглядно такое возрастание можно наблюдать на графиках модуля Фурье-образа
    исходного и фильтрованного сигналов -- на рисунке \ref{fig:fig2} амплитуды частот больше прижимаются к линии 
    $A=0$, а на рисунке \ref{fig:fig8} они больше стремятся к линии $A=50$, где $A$ -- амплитуда. Значение
    параметра $b$ влияет на эффективность фильтрации -- чем меньше значение $b$, тем чище сигнал и тем лучше
    фильтрация. При большом количестве белого шума фильтрация ухудшается (см. рис. \ref{fig:fig7}).
    При $b=0$ получается особый случай -- сигнал превращается в прямоугольную функцию, а фильтрованный
    сигнал выглядит как ее аппроксимация.


    Больше всего влияния на эффективность фильтрации оказывает частота среза $\nu_0$. Этот параметр необходимо подобрать
    так, чтобы оставались только те частоты, которые имеют заметно более высокую амплитуду по сравнению с остальными.
    Чем меньше частота среза $\nu_0$, тем заметно лучше фильтрация. Однако если взять слишком маленькое значение $\nu_0$,
    то фильтрация будет слишком сильной, и мы потеряем большую часть значащих частот в сигнале, что можно увидеть на рисунках
    \ref{fig:fig11}, \ref{fig:fig12}. Это происходит потому, что мы задели и нижние частоты тоже, оставив лишь малую их часть. Можно сказать, что
    $\nu_0$ -- это порог, которым мы определяем, является ли частота верхней или нижней. Такой параметр нужно подбирать с умом, чтобы
    получить от фильтрации ожидаемый результат. Если взять слишком больше значение $\nu_0$, то фильтрация будет не очень эффективной
    (см. рис. \ref{fig:fig25}, \ref{fig:fig26}), так как мы оставили некоторые верхние частоты, которые можно было обнулить.


    \subsection{Убираем специфические частоты. Режекторный фильтр}
    Возьмем ненулевые параметры $b,\,c,\,d$. Попробуем
    обнулять некоторые диапазоны частот, а также
    совместим различные варианты фильтрации, чтобы по возможности убрать влияние обеих компонент помехи.
    Исследуем влияние частот среза и значений параметров $b,\,c,\,d$ на вид помехи и эффективность
    фильтрации. Отдельно рассмотрим случай для $b=0$.


    Синусоидальный шум отображается на графике Фурье-образа как высокий пик помимо исходного, который мы наблюдали ранее в диапазоне низких частот
    -- там содержится наибольшее количество информации об исходном сигнале. Обнулим этот высокий дополнительный пик и посмотрим результат. Должен
    получиться сигнал, похожий на прямоугольную функцию. Также совместим фильтрацию специфических частот с низкими, то есть уберем высокие, которые
    несут наименьшее количество информации об исходном сигнале, но вносят значительный шум.


    Далее будут приведены рисунки полученных графиков. На каждом графике подписаны выбранные значения $b,\,c,\,d,\,\nu_0$. 
    Синим цветом обозначается оригинальный сигнал, красным фильтрованный. 

    
    После анализа графиков можно сделать вывод, что параметр $c$ отвечает за похожесть сигнала на синусоиду с разрывом
    (см. рис. \ref{fig:fig71}, \ref{fig:fig01}). Чем больше значение параметра $c$,
    тем больше растягивается сигнал по оси $Oy$, то есть сильнее возрастает амплитуда волн. При $b=0$ сигнал $u$ описывается
    в формулой синусоиды $$u=g+c\cdot\sin{(d\cdot t+0)}$$ (см. рис. \ref{fig:fig81}, \ref{fig:fig91}). Чем больше $g$, тем выше поднимается график.
    Параметр $d$ характеризует растяжение графика по оси $Ox$. При увеличении значения параметра $d$ частота волн повышается.
    Значение параметра $b$ как и в пунктах ранее влияет на загрязненность сигнала -- чем меньше $b$, тем чище исходный и фильтрованный сигналы.
    Как виды помехи параметры $b,\,c,\,d$ можно назвать амплитудным, наклонным/дуговым (синусоидальным) и частотным шумами соответственно.
    Исследовав влияние частот среза можно сказать, что обнулять частоты пиков слева и справа от наивысшей(их) амплитуд(ы) недостаточно. Такой
    подход скорее искривляет сигнал (см. рис. \ref{fig:fig71}). При обнулении соседних пиков вместе (расширении диапазона обнуляемых
    частот) фильтрация снова скорее искажает исходный сигнал (см. рис. \ref{fig:dklfksdjfo}, \ref{fig:vbvbv}). Фильтр верхних частот выравнивает исходный сигнал,
    что мы уже наблюдали в пункте ранее. Совместно с фильтром специфических частот интересных результатов не обнаружилось. Лучше всего справился фильтр нижних
    частот.
    Вместе с фильтром специфических частот выходит достаточно интересный результат, скорее всего второй лучший из всех (см. рис. \ref{fig:fig77}, \ref{fig:fig87},
    \ref{fig:fig97}, \ref{fig:fig107}, \ref{fig:fig07}, \ref{fig:sdjfsdfj}, \ref{fig:fsdjflsfl}, \ref{fig:snf}).


    \begin{figure}[!htb]
        \centering
        \includegraphics[scale=0.48]{1_u_flt_u_nospec.png}
        \captionsetup{skip=0pt}
        \caption{График исходного и фильтрованного сигналов (1)}
        \label{fig:fig71}
    \end{figure}
    \begin{figure}[!htb]
        \centering
        \includegraphics[scale=0.48]{1_abs_u_U_nospec.png}
        \captionsetup{skip=0pt}
        \caption{График модуля Фурье-образа исходного и фильтрованного сигналов (1)}
        \label{fig:fig72}
    \end{figure}
    \begin{figure}[!htb]
        \centering
        \includegraphics[scale=0.48]{1_3_u_flt_u_nospec.png}
        \captionsetup{skip=0pt}
        \caption{График исходного и фильтрованного сигналов (1)}
        \label{fig:fig77}
    \end{figure}
    \begin{figure}[!htb]
        \centering
        \includegraphics[scale=0.48]{1_3_abs_u_U_nospec.png}
        \captionsetup{skip=0pt}
        \caption{График модуля Фурье-образа исходного и фильтрованного сигналов (1)}
        \label{fig:fig78}
    \end{figure}
    \begin{figure}[!htb]
        \centering
        \includegraphics[scale=0.48]{2_u_flt_u_nospec.png}
        \captionsetup{skip=0pt}
        \caption{График исходного и фильтрованного сигналов (2)}
        \label{fig:fig81}
    \end{figure}
    \begin{figure}[!htb]
        \centering
        \includegraphics[scale=0.48]{2_abs_u_U_nospec.png}
        \captionsetup{skip=0pt}
        \caption{График модуля Фурье-образа исходного и фильтрованного сигналов (2)}
        \label{fig:fig82}
    \end{figure}
    \begin{figure}[!htb]
        \centering
        \includegraphics[scale=0.48]{2_3_u_flt_u_nospec.png}
        \captionsetup{skip=0pt}
        \caption{График исходного и фильтрованного сигналов (2)}
        \label{fig:fig87}
    \end{figure}
    \begin{figure}[!htb]
        \centering
        \includegraphics[scale=0.48]{2_3_abs_u_U_nospec.png}
        \captionsetup{skip=0pt}
        \caption{График модуля Фурье-образа исходного и фильтрованного сигналов (2)}
        \label{fig:fig88}
    \end{figure}
    \begin{figure}[!htb]
        \centering
        \includegraphics[scale=0.48]{3_u_flt_u_nospec.png}
        \captionsetup{skip=0pt}
        \caption{График исходного и фильтрованного сигналов (3)}
        \label{fig:fig91}
    \end{figure}
    \begin{figure}[!htb]
        \centering
        \includegraphics[scale=0.48]{3_abs_u_U_nospec.png}
        \captionsetup{skip=0pt}
        \caption{График модуля Фурье-образа исходного и фильтрованного сигналов (3)}
        \label{fig:fig92}
    \end{figure}
    \begin{figure}[!htb]
        \centering
        \includegraphics[scale=0.48]{3_3_u_flt_u_nospec.png}
        \captionsetup{skip=0pt}
        \caption{График исходного и фильтрованного сигналов (3)}
        \label{fig:fig97}
    \end{figure}
    \begin{figure}[!htb]
        \centering
        \includegraphics[scale=0.48]{3_3_abs_u_U_nospec.png}
        \captionsetup{skip=0pt}
        \caption{График модуля Фурье-образа исходного и фильтрованного сигналов (3)}
        \label{fig:fig98}
    \end{figure}
    \begin{figure}[!htb]
        \centering
        \includegraphics[scale=0.48]{4_u_flt_u_nospec.png}
        \captionsetup{skip=0pt}
        \caption{График исходного и фильтрованного сигналов (4)}
        \label{fig:fig101}
    \end{figure}
    \begin{figure}[!htb]
        \centering
        \includegraphics[scale=0.48]{4_abs_u_U_nospec.png}
        \captionsetup{skip=0pt}
        \caption{График модуля Фурье-образа исходного и фильтрованного сигналов (4)}
        \label{fig:fig102}
    \end{figure}
    \begin{figure}[!htb]
        \centering
        \includegraphics[scale=0.48]{4_3_u_flt_u_nospec.png}
        \captionsetup{skip=0pt}
        \caption{График исходного и фильтрованного сигналов (4)}
        \label{fig:fig107}
    \end{figure}
    \begin{figure}[!htb]
        \centering
        \includegraphics[scale=0.48]{4_3_abs_u_U_nospec.png}
        \captionsetup{skip=0pt}
        \caption{График модуля Фурье-образа исходного и фильтрованного сигналов (4)}
        \label{fig:fig108}
    \end{figure}
    \begin{figure}[!htb]
        \centering
        \includegraphics[scale=0.48]{5_u_flt_u_nospec.png}
        \captionsetup{skip=0pt}
        \caption{График исходного и фильтрованного сигналов (5)}
        \label{fig:fig01}
    \end{figure}
    \begin{figure}[!htb]
        \centering
        \includegraphics[scale=0.48]{5_abs_u_U_nospec.png}
        \captionsetup{skip=0pt}
        \caption{График модуля Фурье-образа исходного и фильтрованного сигналов (5)}
        \label{fig:fig02}
    \end{figure}
    \begin{figure}[H]
        \centering
        \includegraphics[scale=0.48]{5_3_u_flt_u_nospec.png}
        \captionsetup{skip=0pt}
        \caption{График исходного и фильтрованного сигналов (5)}
        \label{fig:fig07}
    \end{figure}
    \begin{figure}[!htb]
        \centering
        \includegraphics[scale=0.48]{5_3_abs_u_U_nospec.png}
        \captionsetup{skip=0pt}
        \caption{График модуля Фурье-образа исходного и фильтрованного сигналов (5)}
        \label{fig:fig08}
    \end{figure}
    \begin{figure}[!htb]
        \centering
        \includegraphics[scale=0.48]{6_u_flt_u_nospec.png}
        \captionsetup{skip=0pt}
        \caption{График исходного и фильтрованного сигналов (6)}
        \label{fig:mlll}
    \end{figure}
    \begin{figure}[!htb]
        \centering
        \includegraphics[scale=0.48]{6_abs_u_U_nospec.png}
        \captionsetup{skip=0pt}
        \caption{График модуля Фурье-образа исходного и фильтрованного сигналов (6)}
        \label{fig:dchjdhc}
    \end{figure}
    \begin{figure}[!htb]
        \centering
        \includegraphics[scale=0.48]{6_3_u_flt_u_nospec.png}
        \captionsetup{skip=0pt}
        \caption{График исходного и фильтрованного сигналов (6)}
        \label{fig:sdjfsdfj}
    \end{figure}
    \begin{figure}[!htb]
        \centering
        \includegraphics[scale=0.48]{6_3_abs_u_U_nospec.png}
        \captionsetup{skip=0pt}
        \caption{График модуля Фурье-образа исходного и фильтрованного сигналов (6)}
        \label{fig:oirjghofgj}
    \end{figure}
    \begin{figure}[!htb]
        \centering
        \includegraphics[scale=0.48]{7_u_flt_u_nospec.png}
        \captionsetup{skip=0pt}
        \caption{График исходного и фильтрованного сигналов (7)}
        \label{fig:dklfksdjfo}
    \end{figure}
    \begin{figure}[H]
        \centering
        \includegraphics[scale=0.48]{7_abs_u_U_nospec.png}
        \captionsetup{skip=0pt}
        \caption{График модуля Фурье-образа исходного и фильтрованного сигналов (7)}
        \label{fig:dfsjusdf}
    \end{figure}
    \begin{figure}[!htb]
        \centering
        \includegraphics[scale=0.48]{7_3_u_flt_u_nospec.png}
        \captionsetup{skip=0pt}
        \caption{График исходного и фильтрованного сигналов (7)}
        \label{fig:fsdjflsfl}
    \end{figure}
    \begin{figure}[H]
        \centering
        \includegraphics[scale=0.48]{7_3_abs_u_U_nospec.png}
        \captionsetup{skip=0pt}
        \caption{График модуля Фурье-образа исходного и фильтрованного сигналов (7)}
        \label{fig:f939}
    \end{figure}
    \begin{figure}[H]
        \centering
        \includegraphics[scale=0.48]{8_u_flt_u_nospec.png}
        \captionsetup{skip=0pt}
        \caption{График исходного и фильтрованного сигналов (8)}
        \label{fig:vbvbv}
    \end{figure}
    \begin{figure}[!htb]
        \centering
        \includegraphics[scale=0.48]{8_abs_u_U_nospec.png}
        \captionsetup{skip=0pt}
        \caption{График модуля Фурье-образа исходного и фильтрованного сигналов (8)}
        \label{fig:wewew}
    \end{figure}
    \begin{figure}[H]
        \centering
        \includegraphics[scale=0.48]{8_3_u_flt_u_nospec.png}
        \captionsetup{skip=0pt}
        \caption{График исходного и фильтрованного сигналов (8)}
        \label{fig:snf}
    \end{figure}
    \begin{figure}[!htb]
        \centering
        \includegraphics[scale=0.48]{8_3_abs_u_U_nospec.png}
        \captionsetup{skip=0pt}
        \caption{График модуля Фурье-образа исходного и фильтрованного сигналов (8)}
        \label{fig:wkljkf}
    \end{figure}


    \subsection{Убираем низкие частоты. Фильтр верхних частот.}
    Рассмотрим графики, где в некоторой окресности точки $\nu=0$ обнулим все значения частот Фурье-образа.
    Такое поведение соответствует фильтру \textit{верхних} частот, так как он пропускает все частоты выше частоты среза.
    Окресность будет настраиваться выбором диапазона частот $[-\nu_0,\nu_0]$. Для лучшей показательности выбраны как маленькие
    окресности около нуля, так и большие. Также рассмотрим поведение фильтрации при различных параметрах $b,\,c,\,d$.


    Далее будут приведены рисунки полученных графиков. На каждом графике подписаны выбранные значения $b,\,c,\,d,\,\nu_0$. 
    Синим цветом обозначается оригинальный сигнал, красным фильтрованный.


    По полученным графикам можно сделать вывод, что фильтр верхних частот выравнивает сигнал так, что линия при амплитуде равной
    нулю становится его серединой. Если брать маленький диапазон $[-\nu_0,\nu_0]$, то сигнал после фильтрации сохраняет свою
    изначальную форму или по крайней мере очень похож на исходный при условии, что исходный сигнал напоминает синусоиду с разрывом
    (имеет хотя бы один полный период синусоиды до разрыва или в разрыве или после разрыва). При этом любые резкие возрастания или
    убывания амплитуд горизонтально центрируются, что можно рассмотреть на рисунках \ref{fig:fig63}, \ref{fig:fig65}, \ref{fig:fig_c} и \ref{fig:fig_e}.
    Явление постоянного амплитудного смещения называется DC offset и является нежелательным, например, в звукозаписи, так как является
    причиной возникновения щелчков в ней. Описанная в предыдущих предложениях фильтрация частично нивелирует данное явление. В остальных
    случаях фильтрованный сигнал концентрируется в некоторой окресности линии $A=0$, где $A$ -- амплитуда. Ровные (прямые) части сигнала становятся
    больше похожи на синусоиду (см. рис. \ref{fig:fig33}, \ref{fig:fig57}, \ref{fig:fig59}, \ref{fig:fig67}), дуговые части отражаются относительно
    $A=0$ и их амплитуды вытягиваются (см. рис. \ref{fig:fig61}, \ref{fig:fig_a}, \ref{fig:fig_g}). При увеличении частотного диапазона $[-\nu_0,\nu_0]$
    синусоиды ровных частей исходного сигнала постепенно превращаются в похожие на исходные ровные (прямые) части, но центрированные относительно линии $A=0$
    (см. рис. \ref{fig:fig45}, \ref{fig:fig47}, \ref{fig:fig49}, \ref{fig:fig51}, \ref{fig:fig53}). Дуговые части испытывают аналогичный процесс (см. рис. \ref{fig:fig_323234}).
    На графиках модуля Фурье-образа видно, как при увеличении диапазона все больше частот с высокими амплитудами отсекаются и, тем ровнее получается
    фильтрованный сигнал.


    \begin{figure}[!htb]
        \centering
        \includegraphics[scale=0.48]{1_u_flt_u_nolow.png}
        \captionsetup{skip=0pt}
        \caption{График исходного и фильтрованного сигналов (1)}
        \label{fig:fig27}
    \end{figure}
    \newpage
    \begin{figure}[!htb]
        \centering
        \includegraphics[scale=0.48]{1_abs_u_U_nolow.png}
        \captionsetup{skip=0pt}
        \caption{График модуля Фурье-образа исходного и фильтрованного сигналов (1)}
        \label{fig:fig28}
    \end{figure}
    \begin{figure}[!htb]
        \centering
        \includegraphics[scale=0.48]{2_u_flt_u_nolow.png}
        \captionsetup{skip=0pt}
        \caption{График исходного и фильтрованного сигналов (2)}
        \label{fig:fig29}
    \end{figure}
    \begin{figure}[H]
        \centering
        \includegraphics[scale=0.48]{2_abs_u_U_nolow.png}
        \captionsetup{skip=0pt}
        \caption{График модуля Фурье-образа исходного и фильтрованного сигналов (2)}
        \label{fig:fig30}
    \end{figure}
    \begin{figure}[!htb]
        \centering
        \includegraphics[scale=0.48]{3_u_flt_u_nolow.png}
        \captionsetup{skip=0pt}
        \caption{График исходного и фильтрованного сигналов (3)}
        \label{fig:fig31}
    \end{figure}
    \newpage
    \begin{figure}[!htb]
        \centering
        \includegraphics[scale=0.48]{3_abs_u_U_nolow.png}
        \captionsetup{skip=0pt}
        \caption{График модуля Фурье-образа исходного и фильтрованного сигналов (3)}
        \label{fig:fig32}
    \end{figure}
    \begin{figure}[!htb]
        \centering
        \includegraphics[scale=0.48]{4_u_flt_u_nolow.png}
        \captionsetup{skip=0pt}
        \caption{График исходного и фильтрованного сигналов (4)}
        \label{fig:fig33}
    \end{figure}
    \begin{figure}[!htb]
        \centering
        \includegraphics[scale=0.48]{4_abs_u_U_nolow.png}
        \captionsetup{skip=0pt}
        \caption{График модуля Фурье-образа исходного и фильтрованного сигналов (4)}
        \label{fig:fig34}
    \end{figure}
    \newpage
    \begin{figure}[!htb]
        \centering
        \includegraphics[scale=0.48]{5_u_flt_u_nolow.png}
        \captionsetup{skip=0pt}
        \caption{График исходного и фильтрованного сигналов (5)}
        \label{fig:fig35}
    \end{figure}
    \begin{figure}[!htb]
        \centering
        \includegraphics[scale=0.48]{5_abs_u_U_nolow.png}
        \captionsetup{skip=0pt}
        \caption{График модуля Фурье-образа исходного и фильтрованного сигналов (5)}
        \label{fig:fig36}
    \end{figure}
    \begin{figure}[!htb]
        \centering
        \includegraphics[scale=0.48]{6_u_flt_u_nolow.png}
        \captionsetup{skip=0pt}
        \caption{График исходного и фильтрованного сигналов (6)}
        \label{fig:fig37}
    \end{figure}
    \newpage
    \begin{figure}[!htb]
        \centering
        \includegraphics[scale=0.48]{6_abs_u_U_nolow.png}
        \captionsetup{skip=0pt}
        \caption{График модуля Фурье-образа исходного и фильтрованного сигналов (6)}
        \label{fig:fig38}
    \end{figure}
    \begin{figure}[!htb]
        \centering
        \includegraphics[scale=0.48]{7_u_flt_u_nolow.png}
        \captionsetup{skip=0pt}
        \caption{График исходного и фильтрованного сигналов (7)}
        \label{fig:fig39}
    \end{figure}
    \begin{figure}[!htb]
        \centering
        \includegraphics[scale=0.48]{7_abs_u_U_nolow.png}
        \captionsetup{skip=0pt}
        \caption{График модуля Фурье-образа исходного и фильтрованного сигналов (7)}
        \label{fig:fig40}
    \end{figure}
    \newpage
    \begin{figure}[!htb]
        \centering
        \includegraphics[scale=0.48]{8_u_flt_u_nolow.png}
        \captionsetup{skip=0pt}
        \caption{График исходного и фильтрованного сигналов (8)}
        \label{fig:fig41}
    \end{figure}
    \begin{figure}[!htb]
        \centering
        \includegraphics[scale=0.48]{8_abs_u_U_nolow.png}
        \captionsetup{skip=0pt}
        \caption{График модуля Фурье-образа исходного и фильтрованного сигналов (8)}
        \label{fig:fig42}
    \end{figure}
    \begin{figure}[!htb]
        \centering
        \includegraphics[scale=0.48]{9_u_flt_u_nolow.png}
        \captionsetup{skip=0pt}
        \caption{График исходного и фильтрованного сигналов (9)}
        \label{fig:fig43}
    \end{figure}
    \newpage
    \begin{figure}[!htb]
        \centering
        \includegraphics[scale=0.48]{9_abs_u_U_nolow.png}
        \captionsetup{skip=0pt}
        \caption{График модуля Фурье-образа исходного и фильтрованного сигналов (9)}
        \label{fig:fig44}
    \end{figure}
    \begin{figure}[!htb]
        \centering
        \includegraphics[scale=0.48]{10_u_flt_u_nolow.png}
        \captionsetup{skip=0pt}
        \caption{График исходного и фильтрованного сигналов (10)}
        \label{fig:fig45}
    \end{figure}
    \begin{figure}[!htb]
        \centering
        \includegraphics[scale=0.48]{10_abs_u_U_nolow.png}
        \captionsetup{skip=0pt}
        \caption{График модуля Фурье-образа исходного и фильтрованного сигналов (10)}
        \label{fig:fig46}
    \end{figure}
    \newpage
    \begin{figure}[!htb]
        \centering
        \includegraphics[scale=0.48]{11_u_flt_u_nolow.png}
        \captionsetup{skip=0pt}
        \caption{График исходного и фильтрованного сигналов (11)}
        \label{fig:fig47}
    \end{figure}
    \begin{figure}[!htb]
        \centering
        \includegraphics[scale=0.48]{11_abs_u_U_nolow.png}
        \captionsetup{skip=0pt}
        \caption{График модуля Фурье-образа исходного и фильтрованного сигналов (11)}
        \label{fig:fig48}
    \end{figure}
    \begin{figure}[!htb]
        \centering
        \includegraphics[scale=0.48]{12_u_flt_u_nolow.png}
        \captionsetup{skip=0pt}
        \caption{График исходного и фильтрованного сигналов (12)}
        \label{fig:fig49}
    \end{figure}
    \newpage
    \begin{figure}[!htb]
        \centering
        \includegraphics[scale=0.48]{12_abs_u_U_nolow.png}
        \captionsetup{skip=0pt}
        \caption{График модуля Фурье-образа исходного и фильтрованного сигналов (12)}
        \label{fig:fig50}
    \end{figure}
    \begin{figure}[!htb]
        \centering
        \includegraphics[scale=0.48]{13_u_flt_u_nolow.png}
        \captionsetup{skip=0pt}
        \caption{График исходного и фильтрованного сигналов (13)}
        \label{fig:fig51}
    \end{figure}
    \begin{figure}[!htb]
        \centering
        \includegraphics[scale=0.48]{13_abs_u_U_nolow.png}
        \captionsetup{skip=0pt}
        \caption{График модуля Фурье-образа исходного и фильтрованного сигналов (13)}
        \label{fig:fig52}
    \end{figure}
    \newpage
    \begin{figure}[!htb]
        \centering
        \includegraphics[scale=0.48]{14_u_flt_u_nolow.png}
        \captionsetup{skip=0pt}
        \caption{График исходного и фильтрованного сигналов (14)}
        \label{fig:fig53}
    \end{figure}
    \begin{figure}[!htb]
        \centering
        \includegraphics[scale=0.48]{14_abs_u_U_nolow.png}
        \captionsetup{skip=0pt}
        \caption{График модуля Фурье-образа исходного и фильтрованного сигналов (14)}
        \label{fig:fig54}
    \end{figure}
    \begin{figure}[!htb]
        \centering
        \includegraphics[scale=0.48]{15_u_flt_u_nolow.png}
        \captionsetup{skip=0pt}
        \caption{График исходного и фильтрованного сигналов (15)}
        \label{fig:fig55}
    \end{figure}
    \newpage
    \begin{figure}[!htb]
        \centering
        \includegraphics[scale=0.48]{15_abs_u_U_nolow.png}
        \captionsetup{skip=0pt}
        \caption{График модуля Фурье-образа исходного и фильтрованного сигналов (15)}
        \label{fig:fig56}
    \end{figure}
    \begin{figure}[!htb]
        \centering
        \includegraphics[scale=0.48]{16_u_flt_u_nolow.png}
        \captionsetup{skip=0pt}
        \caption{График исходного и фильтрованного сигналов (16)}
        \label{fig:fig57}
    \end{figure}
    \begin{figure}[!htb]
        \centering
        \includegraphics[scale=0.48]{16_abs_u_U_nolow.png}
        \captionsetup{skip=0pt}
        \caption{График модуля Фурье-образа исходного и фильтрованного сигналов (16)}
        \label{fig:fig58}
    \end{figure}
    \newpage
    \begin{figure}[!htb]
        \centering
        \includegraphics[scale=0.48]{17_u_flt_u_nolow.png}
        \captionsetup{skip=0pt}
        \caption{График исходного и фильтрованного сигналов (17)}
        \label{fig:fig59}
    \end{figure}
    \begin{figure}[!htb]
        \centering
        \includegraphics[scale=0.48]{17_abs_u_U_nolow.png}
        \captionsetup{skip=0pt}
        \caption{График модуля Фурье-образа исходного и фильтрованного сигналов (17)}
        \label{fig:fig60}
    \end{figure}
    \begin{figure}[!htb]
        \centering
        \includegraphics[scale=0.48]{18_u_flt_u_nolow.png}
        \captionsetup{skip=0pt}
        \caption{График исходного и фильтрованного сигналов (18)}
        \label{fig:fig61}
    \end{figure}
    \newpage
    \begin{figure}[!htb]
        \centering
        \includegraphics[scale=0.48]{18_abs_u_U_nolow.png}
        \captionsetup{skip=0pt}
        \caption{График модуля Фурье-образа исходного и фильтрованного сигналов (18)}
        \label{fig:fig62}
    \end{figure}
    \begin{figure}[!htb]
        \centering
        \includegraphics[scale=0.48]{19_u_flt_u_nolow.png}
        \captionsetup{skip=0pt}
        \caption{График исходного и фильтрованного сигналов (19)}
        \label{fig:fig63}
    \end{figure}
    \begin{figure}[!htb]
        \centering
        \includegraphics[scale=0.48]{19_abs_u_U_nolow.png}
        \captionsetup{skip=0pt}
        \caption{График модуля Фурье-образа исходного и фильтрованного сигналов (19)}
        \label{fig:fig64}
    \end{figure}
    \newpage
    \begin{figure}[!htb]
        \centering
        \includegraphics[scale=0.48]{20_u_flt_u_nolow.png}
        \captionsetup{skip=0pt}
        \caption{График исходного и фильтрованного сигналов (20)}
        \label{fig:fig65}
    \end{figure}
    \begin{figure}[!htb]
        \centering
        \includegraphics[scale=0.48]{20_abs_u_U_nolow.png}
        \captionsetup{skip=0pt}
        \caption{График модуля Фурье-образа исходного и фильтрованного сигналов (20)}
        \label{fig:fig66}
    \end{figure}
    \begin{figure}[!htb]
        \centering
        \includegraphics[scale=0.48]{21_u_flt_u_nolow.png}
        \captionsetup{skip=0pt}
        \caption{График исходного и фильтрованного сигналов (21)}
        \label{fig:fig67}
    \end{figure}
    \newpage
    \begin{figure}[!htb]
        \centering
        \includegraphics[scale=0.48]{21_abs_u_U_nolow.png}
        \captionsetup{skip=0pt}
        \caption{График модуля Фурье-образа исходного и фильтрованного сигналов (21)}
        \label{fig:fig68}
    \end{figure}
    \begin{figure}[!htb]
        \centering
        \includegraphics[scale=0.48]{22_u_flt_u_nolow.png}
        \captionsetup{skip=0pt}
        \caption{График исходного и фильтрованного сигналов (22)}
        \label{fig:fig_a}
    \end{figure}
    \begin{figure}[!htb]
        \centering
        \includegraphics[scale=0.48]{22_abs_u_U_nolow.png}
        \captionsetup{skip=0pt}
        \caption{График модуля Фурье-образа исходного и фильтрованного сигналов (22)}
        \label{fig:fig_b}
    \end{figure}
    \newpage
    \begin{figure}[!htb]
        \centering
        \includegraphics[scale=0.48]{23_u_flt_u_nolow.png}
        \captionsetup{skip=0pt}
        \caption{График исходного и фильтрованного сигналов (23)}
        \label{fig:fig_c}
    \end{figure}
    \begin{figure}[!htb]
        \centering
        \includegraphics[scale=0.48]{23_abs_u_U_nolow.png}
        \captionsetup{skip=0pt}
        \caption{График модуля Фурье-образа исходного и фильтрованного сигналов (23)}
        \label{fig:fig_d}
    \end{figure}
    \begin{figure}[!htb]
        \centering
        \includegraphics[scale=0.48]{24_u_flt_u_nolow.png}
        \captionsetup{skip=0pt}
        \caption{График исходного и фильтрованного сигналов (24)}
        \label{fig:fig_e}
    \end{figure}
    \newpage
    \begin{figure}[!htb]
        \centering
        \includegraphics[scale=0.48]{24_abs_u_U_nolow.png}
        \captionsetup{skip=0pt}
        \caption{График модуля Фурье-образа исходного и фильтрованного сигналов (24)}
        \label{fig:fig_f}
    \end{figure}
    \begin{figure}[!htb]
        \centering
        \includegraphics[scale=0.48]{25_u_flt_u_nolow.png}
        \captionsetup{skip=0pt}
        \caption{График исходного и фильтрованного сигналов (25)}
        \label{fig:fig_g}
    \end{figure}
    \begin{figure}[!htb]
        \centering
        \includegraphics[scale=0.48]{25_abs_u_U_nolow.png}
        \captionsetup{skip=0pt}
        \caption{График модуля Фурье-образа исходного и фильтрованного сигналов (25)}
        \label{fig:fig_h}
    \end{figure}
    \begin{figure}[!htb]
        \centering
        \includegraphics[scale=0.48]{26_u_flt_u_nolow.png}
        \captionsetup{skip=0pt}
        \caption{График исходного и фильтрованного сигналов (26)}
        \label{fig:fig_323234}
    \end{figure}
    \begin{figure}[!htb]
        \centering
        \includegraphics[scale=0.48]{26_abs_u_U_nolow.png}
        \captionsetup{skip=0pt}
        \caption{График модуля Фурье-образа исходного и фильтрованного сигналов (26)}
        \label{fig:fig_56456}
    \end{figure}
    \newpage


    \section{Задание 2. Фильтрация звука.}
    В данном задании необходимо убрать шумы из записи голоса в файле MUHA.wav так, чтобы остался только голос. 
    При прослушивании записи голос слышно не очень хорошо, так как присутствует громкий гул. Построим графики
    исходного сигнала аудиозаписи и его Фурье-образа. Определим по второму графику, какие частоты могут создавать
    шумы. Так как гул громкий, нам необходимо вырезать частоты из аудиозаписи, имеющие наибольшую амплитуду. Такие частоты мы можем
    наблюдать примерно в диапазоне $[-300,300]$ Гц. Чтобы вырезать эти частоты, потребуется фильтр верхних частот, который
    мы рассматривали в задании 1. После применения фильтра построим сравнительный график исходного и фильтрованного
    сигналов аудиозаписи. Можем наблюдать, насколько чище стал сигнал -- оставшиеся амплитуды являются голосом, который 
    нам нужен. Это легко понять исходя из того, что между каждым возрастанием амплитуд есть интервал с падением амплитуд, 
    где они находится в некоторой окресности нуля -- это паузы в предложении, которое говорит человек на записи. Также построим
    сравнительный график модуля исходного и фильтрованного сигналов аудиозаписи. На нем мы видим, что мы успешно вырезали
    низкие частоты с наибольшей амплитудой, которые соответствовали громкому гулу. Теперь послушаем аудиозапись filtered\_{MUHA}.wav, которая 
    оставлена на этом \href{https://drive.google.com/drive/folders/1AuXIiKRWvXFOtJqV3uqPzC494zZ7vCrd?usp=sharing}{гугл-диске}, 
    и убедимся в том, что все посторонние шумы пропали. На записи голос слышно хорошо, однако остался некоторый звуковой эффект 
    фейзер. Избавиться от него удалось обрезав частоты в диапазоне $[-22050, -1000]$ и $[1000, 22050]$ Гц, то есть применив фильтр нижних частот,
    однако голос сильно потерял в качестве. Прослушать этот вариант можно на том же
    \href{https://drive.google.com/drive/folders/1AuXIiKRWvXFOtJqV3uqPzC494zZ7vCrd?usp=sharing}{гугл-диске},
    файл выложен под названием filtered\_{MUHA}\_{2}.wav.


    Далее представлены рисунки с графиками, о которых говорилось в предыдущем абзаце. Синим цветом обозначен исходный сигнал,
    красным -- фильтрованный. Также представлены графики для фильтрованной аудиозаписи filtered\_{MUHA}\_{2}.wav. Стоит заметить,
    что на рисунке \ref{fig:fig116} видно, как окресность нуля стала меньше по сравнению с рисунком \ref{fig:fig113}, то есть в 
    подобных диапазонах амплитуды частот уменьшились (убрали высокочастотный эффект фейзер, оставив диапазон с низкими частотами, но
    без тех, что создавали громкий гул).


    \begin{figure}[!htb]
        \centering
        \includegraphics[scale=0.48]{noisy_audio.png}
        \captionsetup{skip=0pt}
        \caption{График исходного сигнала аудиозаписи.}
        \label{fig:fig111}
    \end{figure}
    \begin{figure}[!htb]
        \centering
        \includegraphics[scale=0.48]{U_audio.png}
        \captionsetup{skip=0pt}
        \caption{График Фурье-образа исходного сигнала аудиозаписи.}
        \label{fig:fig112}
    \end{figure}
    \begin{figure}[!htb]
        \centering
        \includegraphics[scale=0.48]{u_flt_u_audio.png}
        \captionsetup{skip=0pt}
        \caption{График исходного и фильтрованного сигналов аудиозаписи (1).}
        \label{fig:fig113}
    \end{figure}
    \begin{figure}[!htb]
        \centering
        \includegraphics[scale=0.48]{abs_u_U_audio.png}
        \captionsetup{skip=0pt}
        \caption{График модуля Фурье-образа исходного и фильтрованного сигналов аудиозаписи (1).}
        \label{fig:fig114}
    \end{figure}
    \begin{figure}[!htb]
        \centering
        \includegraphics[scale=0.48]{u_flt_u_audio_v2.png}
        \captionsetup{skip=0pt}
        \caption{График исходного и фильтрованного сигналов аудиозаписи (2).}
        \label{fig:fig115}
    \end{figure}
    \begin{figure}[H]
        \centering
        \includegraphics[scale=0.48]{abs_u_U_audio_v2.png}
        \captionsetup{skip=0pt}
        \caption{График модуля Фурье-образа исходного и фильтрованного сигналов аудиозаписи (2).}
        \label{fig:fig116}
    \end{figure}

    
    \section{Листинги программных реализаций}
    В этой секции будут рассмотрены программные реализации, которые использовались по ходу выполнения лабораторной работы.
    Программы написаны на языке python с подключенными библиотеками numpy и matplotlib.


    Два файла, которые необходимы для задания массива времени и частот, вычисления массива функций $g$, задания
    сигнала $u$, а также подсчета Фурье-образа сигнала $u$. Достаточно передать в функции необходимые параметры и
    далее их результат можно использовать для выполнения фильтрации и построения графиков.


    \begin{lstlisting}[label=l1, caption={Файл help.py. Вспомогательные функции.}]
    def get_t(T, dt):
        return np.arange(-T/2, T/2 + dt, dt)   
        
    def get_v(V, dv):
        return np.arange(-V/2, V/2 + dv, dv)
          
    def get_U(u):
        return np.fft.fftshift(np.fft.fft(u))
        
    def g_f(t, t_1, t_2, a):
        if (t_1 <= t <= t_2):
            return a
        return 0
        
    def u_f(g_fs: list, time: list, b, c, d):
        return np.array(g_fs) + \
               b*(np.random.rand(len(time))-0.5) + \
               c*np.sin(d*time)
        
    def get_g_fs(time: list, t_1, t_2, a):
        gs = []
        for t in range(len(time)):
            gs.append(g_f(time[t], t_1, t_2, a))
        
        return gs   
    \end{lstlisting}
    \begin{lstlisting}[label=l2, caption={Файл static.py. Вспомогательные переменные.}]
    a = 2
    t_1 = -1.5
    t_2 = 2.5
        
    T = 10
    dt = 0.01
        
    V = 1/dt
    dv = 1/T
        
    time = hp.get_t(T, dt)
    freq = hp.get_v(V, dv)
    g_fs = hp.get_g_fs(time, t_1, t_2, a)
    \end{lstlisting}


    Программа для построения сравнительных графиков исходного и фильтрованного сигналов (функция build\_{u}\_\_{flt}\_{u}), 
    модуля Фурье-образа исходного и фильтрованного сигналов (функции build\_{abs}\_{U}\_\_{flt}\_{U} и build\_{abs}\_{u}\_{to}\_{U}\_\_{flt}\_{U}, 
    где вторая из приведенных в данном абзаце -- вспомогательная. В нее можно передать исходный сигнал $u$, который преобразуется в Фурье-образ.
    Написана лишь для удобства пользователя) и графиков исходного сигнала или его Фурье-образа (функции build\_{u}\_{or}\_{U} и build\_{u}\_{to}\_{U}, 
    где вторая из приведенных в данном абзаце нужна для удобства). Все функции принимают максимально много необязательных параметров для гибкой настройки
    графика.


    \begin{lstlisting}[label=l3, caption={Файл builder.py. Реализация построения графиков.}]
    def build_u_to_U(freq: list, u: list, clr='b',
                xl1=None, xl2=None, yl1=None,
                yl2=None, xlab='Frequency', ylab='Amplitude',
                label=None, title=None, fz1=6.4, 
                fz2=4.8, legend: bool = True, grid: bool = True):
        U = get_U(u)
        build_u_or_U(freq, U, clr,
                    xl1, xl2, yl1,
                    yl2, xlab, ylab,
                    label, title, fz1, 
                    fz2, legend, grid)

    def build_u_or_U(torv: list, uorU: list, clr='b',
                xl1=None, xl2=None, yl1=None,
                yl2=None, xlab=None, ylab='Amplitude',
                label=None, title=None, fz1=6.4, 
                fz2=4.8, legend: bool = True, grid: bool = True):
        plt.plot(torv, uorU, color=clr, label=label)
        plt.xlabel(xlab)
        plt.ylabel(ylab)
        plt.xlim(xl1, xl2)
        plt.ylim(yl1, yl2)
        plt.title(title)
        if legend:
            plt.legend()
        plt.grid(grid)
        plt.gcf().set_size_inches(fz1, fz2)
        plt.show()

    def build_u__flt_u(time: list, u: list, flt_u: list,
                clr1='b', clr2='r', lab1='Noisy signal',
                lab2='Filtered signal', xlab='Time', ylab='Amplitude',
                title=None, fz1=6.4, fz2=4.8, 
                legend: bool = True, grid: bool = True, xl1=None, 
                xl2=None, yl1=None, yl2=None):
        plt.plot(time, u, color=clr1, label=lab1)
        plt.plot(time, flt_u, color=clr2, label=lab2)
        plt.xlabel(xlab)
        plt.ylabel(ylab)
        plt.xlim(xl1, xl2)
        plt.ylim(yl1, yl2)
        plt.title(title)
        if legend:
            plt.legend()
        plt.grid(grid)
        plt.gcf().set_size_inches(fz1, fz2)
        plt.show()

    def build_abs_u_to_U__flt_U(freq: list, u: list, flt_U: list,
                clr1='b', clr2='r', lab1='Abs noisy signal',
                lab2='Abs filtered signal', xlab='Frequency', ylab='Amplitude',
                xl1=None, xl2=None, yl1=None, 
                yl2=None, title=None, fz1=6.4, 
                fz2=4.8, legend: bool = True, grid: bool = True):
        U = get_U(u)
        build_abs_U__flt_U(freq, U, flt_U, 
                    clr1, clr2, lab1, 
                    lab2, xlab, ylab, 
                    xl1, xl2, yl1, 
                    yl2, title, fz1, 
                    fz2, legend, grid)

    def build_abs_U__flt_U(freq: list, U: list, flt_U: list,
                clr1='b', clr2='r', lab1='Abs noisy signal',
                lab2='Abs filtered signal', xlab='Frequency', ylab='Amplitude',
                xl1=None, xl2=None, yl1=None, 
                yl2=None, title=None, fz1=6.4, 
                fz2=4.8, legend: bool = True, grid: bool = True):
        plt.plot(freq, np.abs(U), color=clr1, label=lab1)
        plt.plot(freq, np.abs(flt_U), color=clr2, label=lab2)
        plt.xlabel(xlab)
        plt.ylabel(ylab)
        plt.xlim(xl1, xl2)
        plt.ylim(yl1, yl2)
        plt.title(title)
        if legend:
            plt.legend()
        plt.grid(grid)
        plt.gcf().set_size_inches(fz1, fz2)
        plt.show()
    \end{lstlisting}


    Программа, реализующая фильтрацию. Пользователь может вызвать такие функции, как filter\_{high}, filter\_{low}, filter\_{special}\_{out}/{in}
    для фильтрации верхних, нижних и специфических частот соответственно. Первые две функции принимают некоторый параметр
    $\nu_0$, а последняя принимает список диапазонов с некоторыми $\nu_{0}^{i}$ и $\nu_{0}^{j}$ (где $i,\,j$ -- индексы, не степени).
    Здесь реализован алгоритм, описанный в первом задании -- берется Фурье-образ, фильтруется и преобразуется обратно из частотного
    пространства во временное. Функция filter\_{U} обнуляет частоты, соответствующие конкретному шагу в зависимости от решения фильтров
    проверки high\_{filter}, low\_{filter} и special\_{filter}\_{out}/{in}.


    \begin{lstlisting}[label=l4, caption={Файл filters.py. Реализация фильтров.}]
    def filter_U(u: list, freq: list, v_0, filter):
        flt_U = get_U(u)
        for i in range(len(freq)):
            freq_i = freq[i]
            if filter(freq_i, v_0):
                flt_U[i] = 0
    
        return flt_U
    
    def low_filter(freq, v_0):
        if -v_0 <= freq <= v_0:
            return False
        return True
    
    def high_filter(freq, v_0):
        return not low_filter(freq, v_0)
    
    def special_filter_in(freq, v_0: list):
        for i in range(len(v_0)):
            if v_0[i][0] <= freq <= v_0[i][1]:
                return False
        return True

    def special_filter_out(freq, v_0: list):
        return not special_filter_in(freq, v_0)
    
    def filter_low(freq: list, u: list, v_0):
        if isinstance(v_0, list) or \
                len(freq) <= 0 or \
                len(u) <= 0:
            return None
    
        flt_U = filter_U(u, freq, v_0, low_filter)
        flt_u = np.fft.ifft(np.fft.ifftshift(flt_U))
        return flt_u, flt_U
    
    def filter_high(freq: list, u: list, v_0):
        if isinstance(v_0, list) or \
                len(freq) <= 0 or \
                len(u) <= 0:
            return None
    
        flt_U = filter_U(u, freq, v_0, high_filter)
        flt_u = np.fft.ifft(np.fft.ifftshift(flt_U))
        return flt_u, flt_U
    
    def filter_special_in(freq: list, u: list, v_0: list):
        if not isinstance(v_0, list) or \
                len(v_0) <= 0 or \
                len(freq) <= 0 or \
                len(u) <= 0:
            return None
    
        flt_U = filter_U(u, freq, v_0, special_filter_in)
        flt_u = np.fft.ifft(np.fft.ifftshift(flt_U))
        return flt_u, flt_U

    def filter_special_out(freq: list, u: list, v_0: list):
        if not isinstance(v_0, list) or \
                len(v_0) <= 0 or \
                len(freq) <= 0 or \
                len(u) <= 0:
            return None
    
        flt_U = filter_U(u, freq, v_0, special_filter_out)
        flt_u = np.fft.ifft(np.fft.ifftshift(flt_U))
        return flt_u, flt_U
    \end{lstlisting}


    Далее представлены программы, в которых используются все предыдущие наработки. Типовой алгоритм -- задать
    параметры $b,\,c,\,d$ и некоторый $\nu_0$, далее воспользоваться функциями создания зашумленного сигнала,
    фильтрации нужных частот и построения необходимых графиков. Любые остальные добавления в код нужны лишь
    для удобства, например, когда нужно построить много графиков и сравнивать их друг с другом или анализировать
    различные результаты по типу совмещения нескольких фильтров на одном сигнале. В случае очистки специфических
    частот задается список $\nu_{0}^{i},\,\nu_{0}^{j}$. В каджый файл импортируются numpy как np, static как st,
    help как hp и filters как ft. Для работы с аудиозаписью подключены библиотеки librosa, scipy и playsound.


    \begin{lstlisting}[label=l5, caption={Файл nohigh.py. Фильтрация нижних частот.}]
    time = st.time
    freq = st.freq
    g_fs = st.g_fs
        
    b = 0.5
    c = 0
    d = 0.1
    v_0 = st.V / 10
        
    u = hp.u_f(g_fs, time, b, c, d)
    flt_u, flt_U = ft.filter_low(freq, u, v_0)
        
    bd.build_u__flt_u(time, u, flt_u,
        title=rf'Low frequency filter. b={b}, c={c}, d={d}, $\nu_0$={v_0}',
                      fz1=12, fz2=6)
    bd.build_abs_u_to_U__flt_U(freq, u, flt_U,
    title=rf'Abs low frequency filter. b={b}, c={c}, d={d}, $\nu_0$={v_0}',
                               xl1=-20, xl2=20, fz1=12, fz2=6)
    \end{lstlisting}
    \begin{lstlisting}[label=l6, caption={Файл nolow.py. Фильтрация верхних частот.}]
    time = st.time
    freq = st.freq
    g_fs = st.g_fs
    
    b = 0.5
    c = 1
    d = 0.1
    v_0 = st.V / 10
    
    u = hp.u_f(g_fs, time, b, c, d)
    flt_u, flt_U = ft.filter_high(freq, u, v_0)
    
    bd.build_u__flt_u(time, u, flt_u,
        title=rf'High frequency filter. b={b}, c={c}, d={d}, $\nu_0$={v_0}',
                      fz1=12, fz2=6)
    bd.build_abs_u_to_U__flt_U(freq, u, flt_U,
    title=rf'Abs high frequency filter. b={b}, c={c}, d={d}, $\nu_0$={v_0}',
                               fz1=12, fz2=6, xl1=-20, xl2=20)
    \end{lstlisting}
    \begin{lstlisting}[label=l7, caption={Файл nospec.py. Фильтрация специфических частот.}]
    time = st.time
    freq = st.freq
    g_fs = st.g_fs
        
    b = 1
    c = 2
    d = 1
    v_0 = 0.39
    v_0_1 = 0.39
    v_0_2 = [[-0.77, -0.39], [0.39, 0.77]]
    u = hp.u_f(g_fs, time, b, c, d)
        
    build_u_U = True
    filter_low = False
    filter_high = False
    filter_special_out = True
        
    flt_u, flt_U = None, None
    flt_u_0, flt_U_0 = None, None
    flt_u_1, flt_U_1 = None, None
    flt_u_2, flt_U_2 = None, None
        
    if build_u_U:
        bd.build_u_or_U(time, u, xlab='Time',
                        title=f'Noisy signal. b={b}, c={c}, d={d}',
                        legend=False, fz1=12, fz2=6)
        bd.build_u_to_U(freq, u,
                        title=f'fft noisy signal. b={b}, c={c}, d={d}',
                        legend=False, fz1=12, fz2=6, xl1=-10, xl2=10)
        
    if filter_low:
     flt_u, flt_U = ft.filter_low(freq, u, v_0)
     bd.build_u__flt_u(time, u, flt_u,
     title=rf'Low frequency filter. b={b}, c={c}, d={d}, $\nu_0$={v_0}',
                       fz1=12, fz2=6)
     bd.build_abs_u_to_U__flt_U(freq, u, flt_U,
     title=rf'Abs low frequency filter. b={b}, c={c}, d={d}, $\nu_0$={v_0}',
                                fz1=12, fz2=6, xl1=-25, xl2=25)
        
    if filter_high:
      flt_u_1, flt_U_1 = ft.filter_high(freq, u, v_0_1)
      bd.build_u__flt_u(time, u, flt_u_1,
      title=rf'High frequency filter. b={b}, c={c}, d={d}, $\nu_0$={v_0_1}',
                        fz1=12, fz2=6)
      bd.build_abs_u_to_U__flt_U(freq, u, flt_U_1,
      title=rf'Abs high frequency filter. b={b}, c={c}, d={d}, $\nu_0$={v_0_1}', fz1=12, fz2=6, xl1=-25, xl2=25)
        
    if filter_special_out:
        flt_u_0, flt_U_0 = ft.filter_special_out(freq, u, v_0_2)
        bd.build_u__flt_u(time, u, flt_u_0,
        title=rf'Special frequency filter. b={b}, c={c}, d={d}, $\nu_0$={v_0_2}', fz1=12, fz2=6)
        bd.build_abs_u_to_U__flt_U(freq, u, flt_U_0,
        title=rf'Abs special frequency filter. b={b}, c={c}, d={d}, $\nu_0$={v_0_2}', fz1=12, fz2=6, xl1=-10, xl2=10)
        
    if filter_low and filter_special_out:
        flt_u_2, flt_U_2 = ft.filter_special_out(freq, flt_u, v_0_2)
        bd.build_u__flt_u(time, u, flt_u_2,
         title=rf'Low and special frequency filter. b={b}, c={c}, d={d}, (1) $\nu_0$={v_0}, (2) $\nu_0$={v_0_2}', fz1=12, fz2=6)
        bd.build_abs_u_to_U__flt_U(freq, u, flt_U_2,
         title=rf'Abs low and special frequency filter. b={b}, c={c}, d={d}, (1) $\nu_0$={v_0}, (2) $\nu_0$={v_0_2}', fz1=12, fz2=6,
                                   xl1=-25, xl2=25)
        
    if filter_high and filter_special_out:
       flt_u_2, flt_U_2 = ft.filter_special_out(freq, flt_u_1, v_0_2)
       bd.build_u__flt_u(time, u, flt_u_2,
       title=rf'High and special frequency filter. b={b}, c={c}, d={d}, (1) $\nu_0$={v_0_1}, (2) $\nu_0$={v_0_2}', fz1=12, fz2=6)
       bd.build_abs_u_to_U__flt_U(freq, u, flt_U_2,
       title=rf'Abs high and special frequency filter. b={b}, c={c}, d={d}, (1) $\nu_0$={v_0_1}, (2) $\nu_0$={v_0_2}', fz1=12, fz2=6,
                                  xl1=-25, xl2=25)
    \end{lstlisting}
    \begin{lstlisting}[label=l8, caption={Файл audio.py. Фильтрация шумов в аудиозаписи.}]
    filter_all = True
    build_U = True
        
    title = 'High frequency filter audio'
    title2 = 'Abs high frequency filter audio'
    if filter_all:
        title = 'High and special frequency filter audio'
        title2 = 'Abs high and special frequency filter audio'
        
    src = 'sound/MUHA.wav'
    filename = 'sound/filtered_MUHA_2.wav'
    try:
        audio, rate = librosa.load(src, sr=None)
    except:
        lab = 'fm_lab3/'
        src = lab + src
        filename = lab + filename
        audio, rate = librosa.load(src, sr=None)
        
    dt = 1 / rate
    T = len(audio) * dt
        
    time = np.linspace(0, T, len(audio), endpoint=False)
    freq = np.linspace(-rate / 2, rate / 2, len(audio), endpoint=False)
        
    flt_u, flt_U = ft.filter_high(freq, audio, 300)
    if filter_all:
        flt_u, flt_U = ft.filter_special_out(freq, flt_u,
                                    [[freq[0], -1000], [1000, freq[-1]]])
        flt_u_float = flt_u.real.astype(np.float32)
        
        
    if build_U:
        bd.build_u_or_U(time, audio, xlab='Time',
                        title='Noisy audio signal', fz1=12, fz2=6,
                        legend=False)
        bd.build_u_to_U(freq, audio, title='fft noisy audio signal',
                        legend=False, fz1=12, fz2=6,
                        xl1=-875, xl2=875, yl1=-4000,
                        yl2=4000)
        
        bd.build_u__flt_u(time, audio, flt_u,
                          title=title, fz1=12, fz2=6)
        bd.build_abs_u_to_U__flt_U(freq, audio, flt_U,
                                   title=title2, xl1=-1500, xl2=1500,
                                   yl1=0, yl2=2000, fz1=12, fz2=6)
        
    write(filename, rate, flt_u_float)
    playsound(filename)
    \end{lstlisting}
\end{document}